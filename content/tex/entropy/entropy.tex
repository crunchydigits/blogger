\documentclass[11pt, oneside]{article}   	% use "amsart" instead of "article" for AMSLaTeX format
\usepackage{geometry}                		% See geometry.pdf to learn the layout options. There are lots.
\geometry{letterpaper}                   		% ... or a4paper or a5paper or ... 
%\geometry{landscape}                		% Activate for rotated page geometry
%\usepackage[parfill]{parskip}    		% Activate to begin paragraphs with an empty line rather than an indent
\usepackage{graphicx}				% Use pdf, png, jpg, or eps§ with pdflatex; use eps in DVI mode
								% TeX will automatically convert eps --> pdf in pdflatex		
\usepackage{amssymb}
\usepackage{amsthm}


\newtheorem{definition}{Definition}


\title{Entropy and Information}
\author{Robert deCarvalho}
\date{}							% Activate to display a given date or no date
\begin{document}
\maketitle


\section{Introduction}
Entropy is one of those words that many people use, but few really understand.
This might be because the concept is invoked in seemingly very different
contexts.  For example, your physics teacher might have told you, ``The entropy
of the universe is increasing!'' Or perhaps you encountered the word while
reading about compression. ``This algorithm compressed our file to within
10\% of the entropy limit.''  Or maybe you vaguely remember entropy as something
your chemistry teacher told you about and maybe it had something to do with
reactions, but you can't quite remember.  Well, hopefully by the end of this
chapter you will have a decent grasp of what entropy is and how you can make
practical use of it.

\section{Why Should You Care About Entropy}
Before we get into what entropy is, let's talk a bit about why we even need such
a concept, and how it is useful.
\begin{enumerate}
    \item Working with probability, want to be honest about minimizing what you
        know.
    \item Can measure population diversity.
    \item Help you reason effectively in the face of uncertainty.  Take into
        account what you know, but make sure you are honest about what you don't
        know.
\end{enumerate}


\section{A Working Definition of Entropy}
Our quest to understand entropy will start with a simple definition.
\begin{definition}
    Entropy is the amount of information you do not know.
\end{definition}
That's a pretty short definition, but why would you possibly care about how much
information you don't know.  Shouldn't you care much more about you do know?
Also, what does it even mean to talk about an ``amount of information?'' You
probably have an intuitive sense of what ``information'' means, but how is it
quantified?  Would you be able to identify when you double the information you
have on some topic As a guide to understanding how information can be measured,
we turn to a contrived example that will illustrate how information can be
quantified.


\subsection{The Library of Information}
Imagine you work at a library containing exactly $1,000$ books.  Instead of the
Dewey Decimal System, the books have been given sequential labels from $000$ to
$999$.  The books are arranged onto $100$ shelves holding $10$ books each and
labeled with two digits corresponding the first two digits of all the books it
contains.  So, for example, books $320$ through $329$ are arranged in order on a
shelf labeled $32$.  When patrons enter the library, they present you with slips
of paper containing the three-digit number of the book they'd like to check out.
It is your job to retrieve the correct book from the shelves.

One day a patron arrives and hands you a slip of paper with the number $248$.
Let's pause and consider the information you have at your disposal to retrieve
the desired book.  You know that every book in the library has been labeled with
a number, and that the books are ordered sequentially onto appropriately labeled
shelves.  By looking at the first two digits on the patron's slip of paper, you
can navigate directly to shelf $24$.  By looking at the last digit, $8$, you
know immediately that you want the second to last book on that shelf.  Because
of the way the library has been organized, you can walk directly to the desired
book and retrieve it on the first try.  You have perfect information on where
the book will be, and there is no missing information to impede your search.

Now consider a modified scenario in which you arrive at the library early one
morning only to discover that a prankster has come in overnight and covered all
the books with identical dust jackets.  There is no way to distinguish between
the books because they all look the same.  To further complicate your life, the
miscreant has gone through each shelf and scrambled the order of the books.
Fortunately, he was considerate enough to ensure that each book remained on the
appropriate shelf, but its location on that shelf is now a completely unknown.

Your first patron of the day arrives bearing a ticket with the number $369$.
What information do you have now to aid your search for the book.  Because of
the prank, all you know is that the book is on shelf $36$, but you don't know
where on that shelf it is.  In order to actually find the book, you will need to
remove the dust cover each of each book until you locate book $369$.  You no
longer have perfect information for locating the book.  You're going to have to
make at least one guess, and you may need to make up to ten guesses in order to
retrieve it.  There is now missing information -- you don't know how the books
on shelf $36$ are arranged.

Finally, imagine a third scenario in which the prankster has abandoned all
civility and has completely shuffled all the books in the entire library, moving
books from shelf to shelf in a completely random way.  In this scenario, when a
patron hands you a ticket, the only thing you know is that the book is somewhere
in the library, but you have no idea where.  You have no information to help you
in your search, so your only option is to one by one remove the dust cover on
each book until you find the what you're looking for.

In each of these three scenarios, you have a different amount of information
available to aid in locating a specific book.  Let's lay the groundwork for
determining how much information you have in each case.  To do so we consider
the labeling system.  Each book has a three digit number associated with it.  If
you were to write down the call number of each book in the library on a piece of
paper, you would end up writing $3\times1,000=3,000$ digits.  It turns out that
the number of digits you need to write down to catalog each book is an excellent
way to measure information.  So, for example, if had $10,000$ books in your
library instead of $1,000$, you would end up needing $4\times10,000=40,000$
digits to account for all the books. This means that a sequentially organized
library of $10,000$ books contains $40,000$ digits of information.  Note that
the information I'm talking about here has to do with the location of the books
and that's all.  If we were talking about any other property of the books other
than their labels and locations in the library, then we would have to increase
the scope of what we mean by the word ``information''.

Let's now think about each of the scenarios and determine how many digits of
information you know in each of them.

In the first scenario, the library was perfectly ordered with $1,000$ books, so
with the three-digit labels on each book, you had $3\times1,000=3,000$ digits of
information.  You knew everything you needed to locate any book, so we'll say
you had $0$ digits of missing information.

In the second scenario, the ordering of books on each shelf was scrambled, but
you still knew which of the $100$ shelves contained any book you were looking
for.  So let's do the math. You know the first two digits of all the books on a
given shelf, so thats $2\times10=20$ digits of information you know.  What you
don't know is the last digit of the books, so that's $1\times10=10$ digits of
missing information.  There are $100$ shelves, so in total, you know
$20\times100=2,000$ digits of information and you are missing
$10\times100=1,000$ digits.  Notice that the total amount of
information remains the same.  The effect of the prankster was to remove $1,000$
digits of known information.

Finally, in the third scenario, you had no idea where any book might be located.
You have $0$ digits of information.  You are missing all the information, which
is three digits for each of the $1,000$ books, or $3\times1,000=3,000$ digits.

\subsection{Entropy In The Library of Information}
Let's take the information accounting we did in the last section and apply our
definition of entropy to identify the entropy in each of the three library
scenarios we presented.

In the first scenario, we knew all the information, and no information was
missing, so there was $0$ digits of entropy.  In the second scenario, the amount
of information we didn't know (i.e. the entropy) was $1,000$ digits.  And
finally in our last scenario, all the information was unknown, so there was
$3,000$ digits of entropy.



\end{document}  
