\documentclass[11pt, oneside]{article}   	% use "amsart" instead of "article" for AMSLaTeX format
\usepackage{geometry}                		% See geometry.pdf to learn the layout options. There are lots.
\geometry{letterpaper}                   		% ... or a4paper or a5paper or ... 
%\geometry{landscape}                		% Activate for rotated page geometry
%\usepackage[parfill]{parskip}    		% Activate to begin paragraphs with an empty line rather than an indent
\usepackage{graphicx}				% Use pdf, png, jpg, or eps§ with pdflatex; use eps in DVI mode
								% TeX will automatically convert eps --> pdf in pdflatex		
\usepackage{amssymb}
\usepackage{amsthm}


\newtheorem{definition}{Definition}


\title{Entropy and Information}
\author{Robert deCarvalho}
\date{}							% Activate to display a given date or no date
\begin{document}
\maketitle


\section{Introduction}
Entropy is one of those words that many people use, but few really understand.
This might be because the concept is invoked in seemingly very different
contexts.  For example, your physics teacher might have told you, ``The entropy
of the universe is increasing!'' Or perhaps you encountered the word while
reading about compression. ``This algorithm compressed our file to within
10\% of the entropy limit.''  Or maybe you vaguely remember entropy as something
your chemistry teacher told you about and maybe it had something to do with
reactions, but you can't quite remember.  Well, hopefully by the end of this
chapter you will have a decent grasp of what entropy is and how you can make
practical use of it.


\section{A Working Definition of Entropy}
Our quest to understand entropy will start with a simple definition.
\begin{definition}
    Entropy is the amount of information you do not know.
\end{definition}
That's a pretty short definition, but why would you possibly care about how much
information you don't know.  Shouldn't you care much more about you do know?
Also, what does it even mean to talk about an ``amount of information?'' You
probably have an intuitive sense of what ``information'' means, but how is it
quantified?  How could you tell if you knew precisely $2.73$ times more about a
topic today than you did yesterday?  As a guide to understanding how information
can be measured, we turn to a contrived example that will illustrate how
information can be quantified.


\subsection{The Library of Information}
Imagine you work at a library containing exactly $1000$ books.  Instead of the
Dewey Decimal System, the books have been given sequential labels from $000$ to
$999$.  The books are arranged onto $100$ shelves each holding $10$ books and
labeled with  two digits corresponding the first two digits of all the books
it contains.  When patrons enter the library, they present you with slips of
paper containing the three-digit number of the book they'd like to check out.
It is your job to retrieve the correct book from the shelves.

A patron arrives and hands you a slip of paper with the number $248$. Let's
pause and consider the information you have at your disposal to retrieve the
desired book.  You know that every book in the library has been labeled with a
number, and that the books are ordered sequentially onto appropriately labeled
shelves.  By looking at the first two digits on the patron's slip of paper, you
can navigate directly to shelf $24$.  By looking at the last digit, $8$, you
know immediately that you want the second to last book on that shelf.  Because
of the way the library has been organized, you can walk directly to the desired
book and retrieve it on the first try.  You have perfect information on where
the book will be, and there is no missing information to impede your search.

Now consider a modified scenario in which you arrive at the library early in the
morning only to discover that a prankster has come in overnight and covered all
the books with identical dust jackets.  There is no way to distinguish between
the books because they all look the same.  To further complicate your life, the
little miscreant has gone through each shelf and scrambled the order of the
books.  Fortunately, he was kind enough to ensure that each book remained on the
appropriate shelf, but its location on that shelf is now a complete mystery.
Now your first patron of the day arrives bearing a ticket with the number $369$.
What information do you have now to aid your search for the book.  Because of
the prank all you can now say is that the book is on shelf $36$, but you don't
know where on the shelf it is.  In order to actually find the book, you will
need to remove the dust cover each of each book on shelf $36$ until you locate
book $369$.  You know longer have perfect information for locating the book.
You're going to have to make at least one guess, and you may need to make up to
ten guesses in order to retrieve it.  There is now missing information -- you
don't know how the books on shelf $36$ are arranged.

Now imagine a third scenario in which the prankster has abandoned all civility
and has completely shuffled all the books in the entire library, moving books
from shelf to shelf in a completely random way.  I this scenario, when a patron
hands you a ticket, the only thing you have is that the book is somewhere
in the library, but you have no idea where.  You have no information to help you
locate it, so your only option is to remove the dust cover on each book, one by
one, until you find the desired book.

\end{document}  
